\documentclass{beamer}
\usetheme{Warsaw}
\usecolortheme{crane}


\title{Hierarchical Models}
\author{Brendon J. Brewer}
\institute{Department of Statistics\\
The University of Auckland}
\date{\color{blue}\url{https://www.stat.auckland.ac.nz/~brewer/}}


\linespread{1.3}
\usepackage{minted}
\usepackage[utf8]{inputenc}
\usepackage{dsfont}
\usepackage{hyperref}


\begin{document}


% DO NOT COMPILE THIS FILE DIRECTLY!
% This is included by the other .tex files.

\begin{frame}[t,plain]
\titlepage
\end{frame}


\begin{frame}[t,plain]
\frametitle{Hierarchical models}
\vspace{2em}
Hierarchical models are useful ways of specifying priors in complex situations
with lots of unknown parameters.

\end{frame}



\begin{frame}[t,plain]
\frametitle{An example}
Suppose you want to measure some properties of the (frequency) distribution of
masses of some stars, but your mass measurements contain noise.\vspace{1em}

Let $\{m_1, m_2, ..., m_N\}$ be the true masses. If you could, you'd infer
some parameters from the $m$s, perhaps with the following assumptions:

\begin{align}
m_i | m_{\rm min},\alpha &\sim \textnormal{Pareto}(m_{\rm min}, \alpha) \\
m_{\rm min} &\sim \textnormal{Something} \\
\alpha      &\sim \textnormal{Something}
\end{align}


\end{frame}

\begin{frame}[t,plain]
\frametitle{An example}

\end{frame}






\end{document}

