\documentclass{beamer}
\usepackage[utf8]{inputenc}
\usepackage{palatino}
\usepackage{subfig}
\usepackage{amsmath}
\usepackage{dsfont}
\usepackage{multimedia}
%\usepackage{minted}

\usetheme{Warsaw}
\usecolortheme{crane}
\linespread{1.3}

% www.sharelatex.com/learn/Beamer

\title{Summary}
\author{Brendon J. Brewer}
\institute{Department of Statistics\\
The University of Auckland}
\date{\color{blue}\url{https://www.stat.auckland.ac.nz/~brewer/}}

\begin{document}


% New slide
\begin{frame}[t, fragile]
\frametitle{Summary}

\hspace*{-1em}\begin{itemize}
\item<2-> Probability can be used to describe certainty/plausibility/degrees of implication.
\item<3-> This is known as `Bayesian statistics'
\item<4-> In astronomy, data rarely resolves all questions of interest. Posterior
          distributions describe the remaining uncertainty
\end{itemize}

\begin{itemize}
\item<5-> The Metropolis algorithm and Nested Sampling are general and useful tools
          for carrying out the required computations.
\end{itemize}

\end{frame}

\end{document}


