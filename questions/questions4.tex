\documentclass[a4paper, 12pt]{article}
\usepackage{amsmath}
\usepackage[utf8]{inputenc}
\usepackage{graphicx}
\usepackage[left=2cm, right=2cm, bottom=3cm, top=2cm]{geometry}
\usepackage{natbib}
\usepackage{microtype}
\usepackage{hyperref}

\newcommand{\given}{\,|\,}

\title{Question Set 4 --- Metropolis and Nested Sampling}
\author{}
\date{}

\begin{document}
\maketitle

%\abstract{\noindent Abstract}

% Need this after the abstract
\setlength{\parindent}{0pt}
\setlength{\parskip}{8pt}

\section*{Question 1}
Consider doing basic `linear regression' - fitting of a straight line to data
$\left\{\{(x_i, y_i)\}_{i=1}^N, N\right\}$.
Make the following assumptions. First, the probability distribution for the
data given the parameters:
\begin{align}
y_i | m, b, \sigma &\sim \textnormal{Normal}(mx_i + b, \sigma^2)
\end{align}

This is the assumption of `gaussian noise' around the straight line. Astronomers
often assume $\sigma$ is known and different for each data point, perhaps given
as a third column in the data file (the `error bars'). In this question I'm
assuming a constant $\sigma$ applies to all data points and it is unknown.

Assume the following naive wide priors for the three unknown parameters:
\begin{align}
m &\sim \textnormal{Normal}(0, 1000^2) \\
b &\sim \textnormal{Normal}(0, 1000^2) \\
\ln\sigma &\sim \textnormal{Uniform}(-10, 10)
\end{align}

Follow these steps to implement this model for the Metropolis algorithm:
\begin{enumerate}
\item Copy {\tt transit\_model.py} to a new file called {\tt straightline.py}, and
work on that.
\item Modify {\tt num\_params} and {\tt data} appropriately, and delete
un-needed variables.
\item Rewrite {\tt from\_prior} and {\tt log\_prior}, and modify
{\tt jump\_sizes}, so they reflect the above
prior distributions.
\item Edit {\tt log\_likelihood} to be appropriate for this problem.
\end{enumerate}

Run {\tt plain\_metropolis.py} and look at trace plots for the parameters.
Use the resulting posterior samples to summarise the inferences about
$m$, $b$, and $\sigma$.


\end{document}

